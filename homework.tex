\documentclass[twoside,11pt]{homework}

\usepackage{dsfont}
\usepackage{amsmath}
\usepackage{graphicx}

\coursename{COMS 4771 Machine Learning (Spring 2020)} 

\studname{Angela Wang}     
\studmail{aw3062@columbia.edu}
\collab{Jane Pan, Stan Liao}
\hwNo{1}      
\date{02/21/2020}  
\begin{document}
\maketitle

\section*{Problem 1} 
\subsection*{(i)}
	The likelihood function of $\theta$ is 
	\begin{align*}
		\mathcal{L}(\theta|X):=& \mathbb{P}(X|\theta) \\
		=& \prod_{i=1}^n \mathbb{P}(x_i | \theta) \tag{i.i.d.}
	\end{align*}
	where $\mathbb{P}(x_i | \theta) = \frac{1}{b-a}$ if $x\in [a,b]$, and $\mathbb{P}(x_i | \theta) = 0$ otherwise. 
	The maximum likelihood estimate is the $\theta$ that maximizes $\mathcal{L}(\theta|X)$, given by
	\begin{align*}
		 \text{arg}\max_\theta \mathcal{L}(\theta|X) 
		=  \text{arg}\max_\theta \log \mathcal{L}(\theta|X) 
		= \text{arg}\max_\theta \log \prod_{i=1}^n \mathbb{P}(x_i | \theta)
	\end{align*}
	To maximize the likelihood, we choose $a\leq \min(x_1,\dots,x_n)$, and  $b\geq \max(x_1,\dots,x_n)$,
	\begin{align*}
		& \text{arg}\max_\theta \mathcal{L}(\theta|X) \\
		= & \text{arg}\max_\theta \log \left(\frac{1}{b-a}\right)^n \\
		= & \text{arg} \max_\theta \, n \log\left(\frac{1}{b-a}\right)
	\end{align*}
	To maximize the logarithm, we should minimize $b-a$. Hence $\theta_{\text{ML}} = (a_{\text{ML}} ,b_{\text{ML}} )$, where
	\begin{align*}
		a_{\text{ML}}  &= \min(x_1,\dots,x_n) \\
		b_{\text{ML}}  &= \max(x_1,\dots,x_n)
	\end{align*}
\subsection*{(ii)}
	\begin{proof}
		Define $\eta = g(\theta)$. Suppose $g$ is bijective, then $\exists g^{-1}$ such that 
		\begin{align*}
			\mathcal{L}(\theta|X) = \mathcal{L}( g^{-1}(\eta) |X)
		\end{align*}
		whose maximum value is $\mathcal{L}(\theta_{\text{ML}}|X) $. 
		To maximize $\mathcal{L}( g^{-1}(\eta) |X)$, choose $\eta_{\text{ML}}$
		such that $g^{-1}(\eta_{\text{ML}})=\theta_{\text{ML}}$. We have $\eta_{\text{ML}}=g(\theta_{\text{ML}})$.\\
		\begin{remark} 
			For the case when $g$ is not bijective, we apply the Invertible Function Theorem to find local inverses for $g$.		
		\end{remark}
	\end{proof}
\subsection*{(iii)}
	Let $X_1,\dots, X_n \sim \mathcal{N}(\mu, \sigma^2)$ be i.i.d random variables.
\subsubsection*{(a)}
	\begin{claim}
		$\frac{1}{n} \sum_{i=1}^n X_i $ is a consistent and unbiased estimator for $\mu$.
	\end{claim}
	\begin{proof}
		By Law of Large Numbers, 
		$ \lim\limits_{n\rightarrow \infty} \frac{1}{n} \sum_{i=1}^n X_i = \mu$.\\
		Since $\mathbb{E}[\frac{1}{n} \sum_{i=1}^n X_i]
		=\frac{1}{n} \sum_{i=1}^n \mathbb{E}[X_i]
		=\frac{1}{n} \sum_{i=1}^n \mu
		=\mu
		$, the estimator is unbiased.
	\end{proof}
	\begin{claim}
		$\frac{1}{n} \sum_{i=1}^n X_i + \frac{1}{n} X_1- \frac{1}{n} X_2$ is a consistent and unbiased estimator for $\mu$.
	\end{claim}
\subsubsection*{(b)}	
	\begin{claim}
		$ \frac{1}{n} \sum_{i=1}^n X_i +\frac{1}{n}$ is a biased but consistent estimator for $\mu$.
	\end{claim}
	\begin{proof}
		$ \lim\limits_{n\rightarrow \infty} (\frac{1}{n} \sum_{i=1}^n X_i +\frac{1}{n})= \mu$.
		$\mathbb{E}[\frac{1}{n} \sum_{i=1}^n X_i +\frac{1}{n}]
		=\frac{1}{n} \sum_{i=1}^n \mathbb{E}[X_i] +\frac{1}{n}
		=\mu+\frac{1}{n}.
		$
	\end{proof}
	\begin{claim}
		$ \frac{1}{n} \sum_{i=1}^n X_i +\frac{1}{n^2}$ is a biased but consistent estimator for $\mu$.
	\end{claim}
\subsubsection*{(c)}
	\begin{claim}
		$X_1$ is an unbiased but inconsistent estimator for $\mu$.
	\end{claim}
	\begin{proof}
		$X_1$ is not consistent since its distribution does not become more concentrated around $\mu$ 
		as the sample size increases.
		$\mathbb{E}[X_1] =\mu$ so $X_1$ is an unbiased estimator.
	\end{proof}
	\begin{claim}
		$X_2$ is an unbiased but inconsistent estimator for $\mu$.
	\end{claim}
\subsubsection*{(d)}
	\begin{claim}
		The constant $1$ is a biased and inconsistent estimator for $\mu$.
	\end{claim}
	\begin{claim}
		$1+X_1$ is a biased and inconsistent estimator for $\mu$.
	\end{claim}
	
\section*{Problem 2} 
\subsection*{(i)}
	Let $\Pi$ be the profit. We can write
	\begin{align*}
		\Pi &= \mathds{1}_{D\geq Q} (P-C)Q + \mathds{1}_{D\leq Q} \left[(P-C)D-C(Q-D)\right] \\
		&= -CQ +\mathds{1}_{D\geq Q} PQ+\mathds{1}_{D\leq Q} PD
	\end{align*}
	The expected profit is given by
	\begin{align*}
		\mathbb{E} [\Pi] &= -CQ+\int_Q^\infty PQ\mathbb{P}(D)dD
		+ \int_{-\infty}^Q PD\mathbb{P}(D)dD \\
		&= -CQ+ PQ\int_Q^\infty \mathbb{P}(D)dD
		+ P\int_{-\infty}^Q D\mathbb{P}(D)dD
	\end{align*}
\subsection*{(ii)}
	\begin{proof}
	Taking the derivative,
	\begin{align*}
		\frac{d\mathbb{E} [\Pi]}{dQ} &= -C + P\int_Q^\infty \mathbb{P}(D)dD
		+ PQ \frac{d}{dQ} \int_Q^\infty \mathbb{P}(D)dD
		+ P \frac{d}{dQ}  \int_{-\infty}^Q D\mathbb{P}(D)dD \\
		&= -C + P\int_Q^\infty \mathbb{P}(D)dD
		- PQ \mathbb{P}(Q)
		+ P Q\mathbb{P}(Q) \tag{Fundamental Theorem of Calculus}\\
		&=-C+P\int_Q^\infty \mathbb{P}(D)dD\\
		&=-C +P(1-F(Q))
	\end{align*}
	To maximize profit, we let $\frac{d\mathbb{E} [\Pi]}{dQ} =-C +P(1-F(Q^*)= 0$. Rearranging, we have
	\begin{align*}
		Q^*= F^{-1}\left(1-\frac{C}{P}\right)
	\end{align*}
	\end{proof}

\section*{Problem 3} 
\subsection*{(i)}
	The error rate is the probability of getting false positive or false negatives:
	\begin{align*}
		& \mathbb{P}[f_t(x)\neq y] \\
		=& \mathbb{P}[Y=y_1 | X\leq t] + \mathbb{P}[Y=y_2 | X\geq t] \\
		=& \int_{-\infty}^t  \mathbb{P}[Y=y_1 | X=x] dx + \int_{t}^\infty  \mathbb{P}[Y=y_2 | X=x] dx
	\end{align*}
\subsection*{(ii)}
	Differentiating the error rate, we have
	\begin{align*}
		& \frac{d}{dt} \mathbb{P}[f_t(x)\neq y]  \\
		=& \, \frac{d}{dt}\int_{-\infty}^t  \mathbb{P}[Y=y_1 | X=x] dx - \frac{d}{dt} \int_{\infty}^t  \mathbb{P}[Y=y_2 | X=x] dx \\
		=&\,  \mathbb{P}[Y=y_1 | X=t] - \mathbb{P}[Y=y_2 | X=t] \tag{Foundamental Theorem of Calculus}
	\end{align*}
	To minimize the error rate, we set $\frac{d}{dt} \mathbb{P}[f_t(x)\neq y] = 0$,
	\begin{align*}
		\mathbb{P}[Y=y_1 | X=t] &= \mathbb{P}[Y=y_2 | X=t] \\
		\mathbb{P}[ X=t | Y=y_1 ] \frac{ \mathbb{P}[ Y=y_1 ]}{ \mathbb{P}[ X=t]}
		&= \mathbb{P}[ X=t | Y=y_2 ] \frac{ \mathbb{P}[ Y=y_2 ]}{ \mathbb{P}[ X=t]} \tag{Bayes Rule}\\
		\mathbb{P}[ X=t | Y=y_1 ] \mathbb{P}[ Y=y_1 ] 
		&= \mathbb{P}[ X=t | Y=y_2 ] \mathbb{P}[ Y=y_2 ]
	\end{align*}
\subsection*{(iii)}
	Let the distribution of $\mathbb{P}[X|Y=y_1]$ be $\mathcal{N}(\mu_1,\sigma_1^2)$, 
	and the distribution of $\mathbb{P}[X|Y=y_2]$ be $\mathcal{N}(\mu_2,\sigma_2^2)$.
	By Bayesian Decision Theory, the Bayes error is the overlapped area under $\mathcal{N}(\mu_1,\sigma_1^2)$
	and $\mathcal{N}(\mu_2,\sigma_2^2)$. \\
	Example in which $f_t$ achieves Bayesian error rate:
	\begin{align*}
		\mathbb{P}[X|Y=y_1] &\sim \mathcal{N}(-1, 1)\\
		\mathbb{P}[X|Y=y_2] &\sim \mathcal{N}(1, 1) \\
		t&=0
	\end{align*}
	Example in which $\forall t\in \mathbb{R}$, $f_t$ does not achieve Bayesian error rate:
	\begin{align*}
		\mathbb{P}[X|Y=y_1] &\sim \mathcal{N}(0, 1)\\
		\mathbb{P}[X|Y=y_2] &\sim \mathcal{N}(0, 2) \\
	\end{align*}


\section*{Problem 4} 
\subsection*{(i)}
	\begin{proof}
		Since $M^{T} = (A^{T}A)^{T} = A^{T}A = M$, M is symmetric. 
		For any column vector $z\in \mathbb{R}^d$, $z^TMz = z^TA^TAz = (Az)^T (Az) = (Az)\cdot(Az) \geq 0$, 
		so M is positive semi-definite.
	\end{proof}
\subsection*{(ii)}
	\begin{proof}
		Base case: if $N=1$, then by definition, 
		\begin{align*}
			\beta^{(1)} &= \beta^{(0)}+ \eta  A^{T} (b-A \beta^{(0)}) \\
			&= \eta v - \eta M \beta^{(0)} \\
			&= \eta v 
		\end{align*}
		Assuming the statement $\beta^{(N)} = \eta \sum_{k=0}^{N-1} (I-\eta M)^k v$ is true for $N=n$, then for $N=n+1$ we have
		\begin{align*}
			\beta^{(n+1)} &= (I-\eta M) \beta^{(n)} + \eta v \tag{definition} \\
			&= (I-\eta M) \eta  \sum_{k=0}^{n-1} (I-\eta M)^k v + \eta v \tag{induction hypothesis} \\
			&= \eta  \sum_{k=0}^{n-1} (I-\eta M)^{(k+1)} v + \eta (I-\eta M)^0 v \\
			&= \eta  \sum_{k=1}^{n} (I-\eta M)^{(k)} v + \eta (I-\eta M)^0 v \\
			&= \eta  \sum_{k=0}^{n} (I-\eta M)^{(k)} v
		\end{align*}
	\end{proof}
\subsection*{(iii)}
	Since $M$ is a real symmetric matrix, it has a decomposition $M = P D P^{-1}$, 
	where $P$ is a matrix composed of orthogonal eigenvectors corresponding to distinct eigenvalues,
	and $D:= \text{diag}(\lambda_1,\dots, \lambda_d)$.
	\begin{align*}
		&\eta \sum_{k=0}^{N-1} (I-\eta M)^k \\
		=& \eta \sum_{k=0}^{N-1} \sum_{i=0}^{k} I^{k-i}(-\eta M)^i \\
		=&  \sum_{k=0}^{N-1} \sum_{i=0}^{k} (-1)^i\eta^{i+1} M^{i} \\
		=&  \sum_{k=0}^{N-1} \sum_{i=0}^{k} (-1)^i\eta^{i+1} PD^{i}P^{-1} \\
		=& P  \left(\sum_{k=0}^{N-1} \sum_{i=0}^{k}  (-1)^i\eta^{i+1}  \text{diag}(\lambda_1^i,\dots, \lambda_d^i) \right) P^{-1} \\
		=& P \, \text{diag} \left(\sum_{k=0}^{N-1} \sum_{i=0}^{k}  (-1)^i\eta^{i+1} \lambda_1^i,\dots, 
		\sum_{k=0}^{N-1} \sum_{i=0}^{k}  (-1)^i\eta^{i+1} \lambda_d^i \right)  P^{-1}\\
		=& P \, \text{diag} \left(
			\sum_{i=0}^{N-1} (N-i) (-1)^{i}\eta^{i+1}\lambda^i_1
			\dots, 
			\sum_{i=0}^{N-1} (N-i) (-1)^{i}\eta^{i+1}\lambda^i_d
		\right)  P^{-1}
	\end{align*}
	Since $I$, $M$ are symmetric, $\eta \sum_{k=0}^{N-1} (I-\eta M)^k$ is symmetric, 
	and its eiganvalues $\lambda_1',\dots, \lambda_d'$ are given by
	\begin{align*}
		\lambda_j' = \sum_{i=0}^{N-1} (N-i) (-1)^{i}\eta^{i+1}\lambda^i_j
	\end{align*}
	for any $j=1,\dots,d$.
\subsection*{(iv)}
	\begin{proof}
	We start by deriving a iterative formula for the difference
	\begin{align*}
		& ||\beta^{(N)} - \hat{\beta}||_2^2 \\
		=& \left( (I-\eta M)^N \hat\beta \right)^T \left( (I-\eta M)^N \hat\beta \right) \\
		=& \, \hat\beta^T (I-\eta M)^{2N} \hat\beta \\
		\leq& \, \hat\beta^T (I-2N\eta M) \hat\beta \tag{Taylor's Theorem} \\
		\leq& \, || I-2N\eta M ||_2 \cdot ||\hat{\beta}||_2^2 \\
		\leq & \, (1-2N\eta \lambda_{\text{min}}) ||\hat{\beta}||_2^2 
		\tag{Since $\sigma_{\text{max}}(I-2N\eta M) = 1-2N\eta \sigma_{\text{min}}(M)$} \\
		\leq& \, e^{-2N\eta\lambda_{\text{min}}}||\hat{\beta}||_2^2 
	\end{align*}
	\end{proof}




\section*{Problem 5}
\subsection*{(i)}
	There may be correlated attributes (e.g. neighborhood) that are proxies of the sensitive attribute (e.g. race). 
	Simply removing race does not decrease bias, since neighborhood still acts as a trait in classification of the data. 
	The model therefore fails to maintain fairness across the sensitive attribute.
\subsection*{(ii)}  
	Since $\mathbb{P}_a[\hat{Y}=1]=\mathbb{P}[\hat{Y}=1|a=0]=\mathbb{P}[\hat{Y}=0|a=1]$, the following are equivalent:
	\begin{align*}
		\mathbb{P}[\hat{Y}=1]&=\mathbb{P}_a[\hat{Y}=1]\\
		\mathbb{P}[\hat{Y}=1|a=0]&=\mathbb{P}[\hat{Y}=0|a=1]\\
		\mathbb{P}_0[\hat{Y}=1]&=\mathbb{P}_1[\hat{Y}=1]
	\end{align*}
\subsection*{(iii)}  
	Now we generalize the statement to $A\in \mathbb{N}$ and $\hat{Y} \in \mathbb{R}$.
	$\forall b_1, b_2  \in A$, $\forall y\in \mathbb{R}$, we have 
	$$\mathbb{P}_{b_1}[\hat{Y}=y] = \mathbb{P}_{b_2}[\hat{Y}=y] 
	\Longleftrightarrow \mathbb{P}[\hat{Y}=y] = \mathbb{P}_a[\hat{Y}=y] \qquad \forall a\in A$$
\subsection*{(viii)}  
	No, there is a tradeoff between bias and variance.








\end{document} 
